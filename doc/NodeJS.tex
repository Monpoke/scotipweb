\chapter{NodeJS, the configuration generator}
\section{Overview}

In order to generate the Asterisk configuration, we needed to create an intermediate server to make the conversion from the website database to Asterisk files. Because NodeJS is fast and easy to launch, we chose it to make this intermediate server. \newline

To refresh the Asterisk configuration, this server provides some URL to call with specific parameters. Indeed, it will be executed as an internal web server which only can be reached by the \textbf{Spring application}. Each time a modification is done by the website on the database, the website will call these REST url itself. \newline

When an URL is called on the \textit{NodeJS} server, it will read some informations in the database from the passed informations in query string. From these informations, it will generate a new configuration file and save it on the server. Then, the Asterisk server will be reloaded through some \textit{CLI} commands.

\section{Hapi Server}

\begin{figure}[!ht]
  \caption{HapiJS logo.}
  \centering
    \includegraphics[width=0.5\textwidth]{img/hapijs}
\end{figure}

\textit{Hapi} is a NodeJS framework which allows to create easily a REST API. It provides all the necessary tools for routing, server, logging... And this framework is one of the most used, with \textit{Express}.

\section{REST API}
	
\subsection{Dialplan generation}
URL: \textbf{/switchboards/{\textit{switchboardID}}}
\paragraph{Needed parameters:}
\begin{itemize}
\item {switchboardID} - A valid switchboard ID
\end{itemize}


\paragraph{Algorithm:}
\begin{itemize}
	\item Checks if the switchboard exists.
	\item Loads all modules from the switchboard in the database.
	\item Checks if there is only one \textit{root module}, and no problem with data. If there is a problem, the reloading is aborted.
	\item Configuration generation:
	
		\begin{itemize}
			\item Generates the header: register some variables, call the \textit{NodeJS} server to notify a call.
			\item Generates all keys aliases for the modules. 
			\item Generates the root module alias.
			\item For each module type, generate the associated configuration.
			\item Register the hang up extension, \textbf{h}, to notify the call end.
		\end{itemize}	
	\item Write to right file.
	\item Reload the Asterisk server by \textit{CLI} command.

\end{itemize}

\subsection{Operators}
URL: \textbf{/operators/{\textit{companyId}}}

\subsection{MOH}
URL: \textbf{/moh/{\textit{companyId}}}

\subsection{Queues}
URL: \textbf{/queues/{\textit{switchboardId}}}


\subsection{Calls logs}
URL: \textbf{/calls/new/{\textit{switchboardId}}}

URL: \textbf{/calls/end/{\textit{callId}}}

URL: \textbf{/calls/action/{\textit{callId}}}

URL: \textbf{/calls/variable/{\textit{callId}}}

\subsection{Module recordings}

URL: \textbf{/dialplan/{sid}/module/exists/{\textit{moduleId}}}
URL: \textbf{/dialplan/{sid}/module/updateFile/{\textit{moduleId}}}



