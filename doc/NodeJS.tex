\chapter{NodeJS, the configuration generator}
\section{Overview}

In order to generate the Asterisk configuration, we needed to create an intermediate server to make the conversion from the website database to Asterisk files. Because NodeJS is fast and easy to launch, we chose it to make this intermediate server. \newline

To refresh the Asterisk configuration, this server provides some URL to call with specific parameters. Indeed, it will be executed as an internal web server which only can be reached by the \textbf{Spring application}. Each time a modification is done by the website on the database, the website will call these REST url itself. \newline

When an URL is called on the \textit{NodeJS} server, it will read some informations in the database from the passed informations in query string. From these informations, it will generate a new configuration file and save it on the server. Then, the Asterisk server will be reloaded through some \textit{CLI} commands.

\section{Hapi Server}

\begin{figure}[!ht]
  \caption{HapiJS logo.}
  \centering
    \includegraphics[width=0.5\textwidth]{img/hapijs}
\end{figure}


