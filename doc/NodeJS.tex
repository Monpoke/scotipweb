\chapter{NodeJS, the configuration generator}
\section{Overview}

In order to generate the Asterisk configuration, we needed to create an intermediate server to make the conversion from the website database to Asterisk files. Because NodeJS is fast and easy to launch, we chose it to make this intermediate server. \newline

To refresh the Asterisk configuration, this server provides some URL to call with specific parameters. Indeed, it will be executed as an internal web server which only can be reached by the \textbf{Spring application}. Each time a modification is done by the website on the database, the website will call these REST url itself. \newline

When an URL is called on the \textit{NodeJS} server, it will read some informations in the database from the passed informations in query string. From these informations, it will generate a new configuration file and save it on the server. Then, the Asterisk server will be reloaded through some \textit{CLI} commands.

\section{Hapi Server}

\begin{figure}[H]
  \caption{HapiJS logo.}
  \centering
    \includegraphics[width=0.5\textwidth]{img/hapijs}
\end{figure}

\textit{Hapi} is a NodeJS framework which allows to create easily a REST API. It provides all the necessary tools for routing, server, logging... And this framework is one of the most used, with \textit{Express}.

\section{REST API}
	
\subsection{Dialplan generation}
The following \textit{URL} will be called by the website when the dialplan has been updated or a module added.
\newline

URL: \textbf{/switchboards/{\textit{switchboardID}}}
\paragraph{Needed parameters:}
\begin{itemize}
\item {switchboardID} - A valid switchboard ID
\end{itemize}


\paragraph{Algorithm:}
\begin{itemize}
	\item Checks if the switchboard exists.
	\item Loads all modules from the switchboard in the database.
	\item Checks if there is only one \textit{root module}, and no problem with data. If there is a problem, the reloading is aborted.
	\item Configuration generation:
	
		\begin{itemize}
			\item Generates the header: register some variables, call the \textit{NodeJS} server to notify a call.
			\item Generates all keys aliases for the modules. 
			\item Generates the root module alias.
			\item For each module type, generate the associated configuration.
			\item Register the hang up extension, \textbf{h}, to notify the call end.
		\end{itemize}	
	\item Write to right file.
	\item Reload the Asterisk server by \textit{CLI} command: : \textbf{asterisk -rx "dialplan reload"}.

\end{itemize}

\subsection{Operators}
The following \textit{URL} will be called by the website when an operator has been added.
\newline

URL: \textbf{/operators/{\textit{companyId}}}

\paragraph{Needed parameters:}
\begin{itemize}
\item {companyID} - A valid company ID
\end{itemize}


\paragraph{Algorithm:}
\begin{itemize}
	\item Loads all the operators from the company with provided ID.
	\item Generates the configuration for Asterisk users. The configuration is based on the extension of an user model, with default settings. \textit{(Skype accounts not created)}
	\item Write to right file.
	\item Reload the Asterisk server by \textit{CLI} command.: \textbf{asterisk -rx "sip reload"}.

\end{itemize}



\subsection{MOH}
The following \textit{URL} will be called by the website when a MOH group has been created.
\newline

URL: \textbf{/moh/{\textit{companyId}}}

\paragraph{Needed parameters:}
\begin{itemize}
\item {companyID} - A valid company ID
\end{itemize}


\paragraph{Algorithm:}
\begin{itemize}
	\item Loads all the MOH group from the company with provided ID.
	\item Generates the configuration with the MOH classes. 
	\item Write to right file.
	\item Reload the Asterisk server by \textit{CLI} command: \textbf{asterisk -rx "moh reload"}.

\end{itemize}




\subsection{Queues}
The following \textit{URL} will be called by the website when a queue has been updated.
\newline


URL: \textbf{/queues/{\textit{switchboardId}}}

\paragraph{Needed parameters:}
\begin{itemize}
\item {companyID} - A valid company ID
\end{itemize}


\paragraph{Algorithm:}
\begin{itemize}
	\item Loads all the Queues from the company with provided ID.
	\item For each queue, register the Queue name and make it extends from a default model. Then, load all operators:
	\begin{itemize}
		\item Register all assigned operators to the queue with this line: \textbf{member => SIP/username}.
		\item If it is a Skype account, add \textbf{@skype}.
\end{itemize}	 
	\item Write to right file.
	\item Reload the Asterisk server by \textit{CLI} command: \textbf{asterisk -rx "queue reload all"}.

\end{itemize}





\subsection{Calls logs}
\subsubsection{New call}
The following URL will be called when someone call the switchboard.
\newline

URL: \textbf{/calls/new/{\textit{switchboardId}}}

\paragraph{Needed parameters:}
\begin{itemize}
\item {switchboardID} - A valid switchboard ID
\item \textit{POST} {callerid} - The caller phone number
\end{itemize}


\paragraph{Algorithm:}
\begin{itemize}
	\item Checks a switchboard exists with the provided ID.
	\item Save to database.
	\item Returns a call ID.

\end{itemize}


\subsubsection{End call}
The following \textit{URL} is called when a call is hung up. 
\newline

URL: \textbf{/calls/end/{\textit{callId}}}


\paragraph{Needed parameters:}
\begin{itemize}
\item {callID} - A valid call ID
\item \textit{POST} {end} - Equals to 1
\end{itemize}


\paragraph{Algorithm:}
\begin{itemize}
	\item Checks if a call exists with the provided ID.
	\item Process the duration time of the call and mark it as finished.
	\item Save to database.

\end{itemize}


\subsubsection{Caller has made an action}
The following \textit{URL} will be called when a caller press a key.
\newline

URL: \textbf{/calls/action/{\textit{callId}}}

\paragraph{Needed parameters:}
\begin{itemize}
\item {callID} - A valid call ID
\item \textit{POST} {mod} - Contains the module ID where the caller is
\end{itemize}


\paragraph{Algorithm:}
\begin{itemize}
	\item Checks if a call exists with the provided ID.
	\item Save the module ID 

\end{itemize}


\subsubsection{Caller has entered a variable}
The following \textit{URL} will be called when the caller has entered a number.
\newline

URL: \textbf{/calls/variable/{\textit{callId}}}

\paragraph{Needed parameters:}
\begin{itemize}
\item {callID} - A valid call ID
\item \textit{POST} {varname} - Contains the variable name
\item \textit{POST} {value} - Contains the variable value
\end{itemize}


\paragraph{Algorithm:}
\begin{itemize}
	\item Checks if a call exists with the provided ID.
	\item Save the variable name associated to its value.

\end{itemize}



\subsection{Module recordings}
The following \textit{URL} will be called by Asterisk when a company wants to record new sounds through its phone.
\newline 
 
\subsubsection{Module existence}

URL: \textbf{/dialplan/{sid}/module/exists/{\textit{moduleId}}}

\paragraph{Needed parameters:}
\begin{itemize}
\item {sid} - A valid switchboard ID
\item {moduleId} - A valid module ID associated to the switchboard ID
\end{itemize}


\paragraph{Algorithm:}
\begin{itemize}
	\item Returns \textit{exists} if a module exists with the provided ID and the associated switchboard matches the provided ID.

\end{itemize}

\subsubsection{Update a file}
URL: \textbf{/dialplan/{sid}/module/updateFile/{\textit{moduleId}}}

\paragraph{Needed parameters:}
\begin{itemize}
\item {sid} - A valid switchboard ID
\item {moduleId} - A valid module ID associated to the switchboard ID
\item \textit{POST} {file} - The filename to play
\end{itemize}


\paragraph{Algorithm:}
\begin{itemize}
	\item Checks \textit{exists} if a module exists with the provided ID and the associated switchboard matches the provided ID.
	\item Save the new file in settings of the module.

\end{itemize}


