\chapter{Server configuration}

\section{Security}

To protect our server, we had to set up some security systems. A lot of these attacks consists to login on some sensitive services: \textit{SSH}, \textit{Asterisk}, \textit{FTP}... And our server didn't escape to these attacks. 

\begin{figure}[!ht]
  \caption{SSH attacks}
  \centering
    \includegraphics[width=1\textwidth]{img/sshattacks.png}
\end{figure}

\subsection{Firewall}

For a simple protection, we generated some firewall rules to allows the inbound port we need. For this project, we needed this following ports list:
\begin{itemize}
\item \textit{SIP}: 5060-5082 (TCP/UDP)	
\item \textit{RTP}: 10000-20000 (UDP)
\item \textit{MySQL}: 3306 (TCP)
\item \textit{SSH}: 22 (TCP)
\item \textit{WEB}: 80,443 (TCP)
\item \textit{DNS}: 53 
\item \textit{PING}: ICMP
\end{itemize}

\subsection{fail2ban}
As said previously, our server has been often attacked. To prevent these attacks by brute-force, we installed the software \textit{fail2ban}. Once installed, it begins to read logs files and blocks suspicious IP for few minutes. Highly customizable, it natively supports a lot of services as \textit{SSH}, \textit{FTP} and \textit{Asterisk}! 
When an IP is blocked, it stays blocked for few minutes, then it is removed from the ban list. If this IP is caught again, it is definitively blocked.


\section{A reverse proxy, Nginx}

By default, Tomcat is running on port \textbf{8080}, so it isn't the default HTTP port which is \textbf{80}. Because all ports under \textit{1024} are reserved to \textit{root} user, we chose to not launch the tomcat server as \textit{root} user to prevent some attacks. In this way, we installed a reverse proxy on the port 80. As reverse proxy, we used the \textbf{Nginx} server.
Because \textbf{Nginx} only redirect the incoming traffic through the internal network to the \textbf{Tomcat} server, it is impossible for the web application to access sensitive data in the system.

\section{MySQL, different users}
To improve security, we also created several users on our MySQL server. While the first user, designed for the Spring application is able to create tables, read and write in tables, the second one is fully restricted to some tables. This second user is designed to be used by the \textit{NodeJS} application. Because this application doesn't need all privileges on all tables, we just set up these following permissions. 

\begin{figure}[H]
  \caption{Restricted user.}
  \centering
    \includegraphics[width=1\textwidth]{img/scotipnodespas.png}
\end{figure}

